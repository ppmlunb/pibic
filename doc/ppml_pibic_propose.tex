\documentclass{article}
\usepackage{blindtext}
\usepackage[a4paper, total={6in, 10in}]{geometry}
\usepackage{cite}
\usepackage{graphicx,color}
\usepackage[portuguese]{babel}
\usepackage[utf8]{inputenc}
\usepackage{amsmath}
\usepackage{amssymb}
\usepackage{authblk} % Para personalizar os campos de autores

\begin{document}

\title{Privacy Preserving Machine Learning: Estudo de Técnicas e Prova de Conceito}
\author{Thaís Fernanda de Castro Garcia, 200043722}
\author{Ualiton Ventura da Silva, 202033580}
\author{Maria Eduarda Carvalho Santos, 190092556}
\affil{Universidade de Brasília}
\date{Maio 2023}

\maketitle

\begin{abstract}
A preservação da privacidade tem se tornado uma preocupação crescente na era da informação, onde uma quantidade cada vez maior de dados pessoais é coletada e analisada. Com o avanço da tecnologia e o aumento da demanda por análises de dados, surge a necessidade de desenvolver técnicas eficazes para proteger a privacidade dos indivíduos enquanto se realiza o aprendizado de máquina. Este projeto de iniciação científica propõe um estudo aprofundado sobre as técnicas de Privacy Preserving Machine Learning (PPML) e a implementação de uma prova de conceito para demonstrar a aplicabilidade dessas técnicas.
\end{abstract}

\section{Introdução}
A preservação da privacidade tem se tornado uma preocupação cada vez mais relevante na sociedade atual. Com o advento da era da informação, a coleta e análise de dados pessoais se tornaram onipresentes em diversas áreas, como saúde, finanças e comércio eletrônico \cite{boulemtafes2020review}). No entanto, o uso desses dados sensíveis apresenta riscos consideráveis à privacidade dos indivíduos. Portanto, torna-se essencial desenvolver técnicas que possibilitem a análise de dados sem comprometer a privacidade das pessoas, dessa forma viabilizando um maior data set para treinar inteligências artificiais que necessitam de dados considerados sensíveis enquanto garantindo a privacidade deles \cite{chamikara2020efficient}.

\section{Objetivos}
O objetivo deste projeto de iniciação científica é estudar a área de Privacy Preserving Machine Learning (PPML), que se dedica a encontrar soluções para realizar aprendizado de máquina de forma segura e preservando a privacidade dos dados. Os principais objetivos deste projeto são:

\begin{itemize}
  \item Investigar as principais técnicas de PPML disponíveis na literatura, compreendendo seus fundamentos teóricos e aplicações práticas.
  \item Analisar os desafios e limitações associados ao uso de PPML, como a preservação da precisão dos modelos e o custo computacional.
  \item Identificar e avaliar os algoritmos de PPML mais adequados para diferentes cenários de aplicação, considerando as características dos dados e os requisitos de privacidade.
  \item Realizar uma prova de conceito usando um conjunto de dados de referência, aplicando técnicas de PPML para garantir a privacidade dos dados durante o processo de treinamento do modelo.
  \item Avaliar a eficácia das técnicas de PPML implementadas em termos de preservação da privacidade, precisão dos modelos resultantes e desempenho computacional.
\end{itemize}

\section{Relevância Científica}

Com o avanço da tecnologia e o crescente uso de dados pessoais, surge a necessidade de proteger a privacidade dos indivíduos e preservar a confidencialidade das informações sensíveis. O machine learning, por sua vez, tem sido amplamente utilizado para extrair insights valiosos desses dados. No entanto, muitas vezes, isso requer compartilhamento e processamento de dados sensíveis, o que pode comprometer a privacidade dos usuários. Nesse contexto, o desenvolvimento de técnicas e algoritmos que permitam o treinamento e a inferência de modelos de aprendizado de máquina sem expor dados pessoais é essencial. Portanto, um projeto de iniciação científica nessa área busca contribuir para a criação de soluções robustas e eficientes, permitindo que as organizações possam aproveitar os benefícios do machine learning sem violar a privacidade dos indivíduos \cite{koti2021swift}.

Além disso, a área de privacy preserving machine learning possui implicações científicas de amplo alcance. Ao desenvolver e aprimorar técnicas que permitem a proteção da privacidade durante o processo de treinamento e inferência de modelos, os pesquisadores estão contribuindo para o desenvolvimento de uma ciência de dados mais ética e responsável. Essas técnicas podem ser aplicadas em uma variedade de contextos, desde a análise de dados de saúde, onde a privacidade médica é fundamental, até a análise de dados financeiros e comerciais, onde a confidencialidade das informações é essencial. Além disso, a pesquisa em privacy preserving machine learning promove a confiança entre os usuários e as organizações que utilizam técnicas de aprendizado de máquina, ao garantir que os dados pessoais sejam tratados com segurança e respeito. Portanto, um projeto de iniciação científica nessa área contribui para o avanço da ciência de dados, ao mesmo tempo em que protege os direitos e a privacidade dos indivíduos.

\section{Metodologia}
A metodologia deste projeto envolverá as seguintes etapas:

\begin{enumerate}
\item Revisão bibliográfica: Realizar uma pesquisa abrangente sobre as técnicas de PPML existentes, explorando estudos científicos, artigos de conferências e documentos relevantes. Compreender as vantagens e desvantagens de cada técnica e identificar os desafios associados à sua implementação.
\item Seleção de técnicas: Analisar os algoritmos de PPML mais promissores e adequados para o contexto do projeto, levando em consideração os requisitos de privacidade, a eficiência computacional e a preservação da precisão dos modelos.
\item Implementação da prova de conceito: Desenvolver um ambiente de experimentação que permita a aplicação das técnicas de PPML selecionadas. Utilizar um conjunto de dados de referência e realizar experimentos controlados para avaliar a privacidade, a precisão e o desempenho dos modelos gerados.
\item Avaliação dos resultados: Analisar os resultados obtidos na prova de conceito, comparando a preservação da privacidade, a precisão dos modelos e o desempenho computacional alcançados pelas diferentes técnicas de PPML. Discutir os pontos fortes e limitações das abordagens utilizadas.

\item Documentação e apresentação: Elaborar um relatório detalhado das atividades realizadas, incluindo a revisão bibliográfica, a metodologia adotada, os resultados obtidos e as conclusões alcançadas. Preparar uma apresentação para compartilhar os principais aspectos do projeto e promover a conscientização sobre a importância da privacidade preservada em machine learning.
\end{enumerate}

\section{Resultados e Impacto Esperados}
Espera-se que este projeto de iniciação científica contribua para o avanço do conhecimento na área de Privacy Preserving Machine Learning. Os principais resultados esperados incluem:

\begin{itemize}
\item Um conhecimento aprofundado das técnicas de PPML disponíveis na literatura, incluindo uma compreensão dos fundamentos teóricos e aplicações práticas dessas técnicas.
\item Uma avaliação abrangente dos desafios e limitações associados ao uso de PPML, permitindo uma compreensão mais clara das restrições e considerações práticas ao implementar essas técnicas.
\item A identificação dos algoritmos de PPML mais adequados para diferentes cenários de aplicação, levando em conta as características dos dados e os requisitos de privacidade específicos de cada contexto \cite{9194237}.
\item A implementação de uma prova de conceito bem-sucedida usando um conjunto de dados de referência, com foco na versão PP do Energy-based Flow Classifier (EFC), para garantir a privacidade dos dados durante o processo de treinamento do modelo \cite{souza2022novel}.
\item Uma avaliação abrangente da eficácia das técnicas de PPML implementadas em termos de preservação da privacidade, precisão dos modelos resultantes e desempenho computacional.
\item Um relatório detalhado documentando as atividades realizadas, incluindo a revisão bibliográfica, a metodologia adotada, os resultados obtidos e as conclusões alcançadas.
\item Uma apresentação clara e concisa dos principais aspectos do projeto, destacando os benefícios

e as implicações práticas das técnicas de PPML no contexto da preservação da privacidade em machine learning.
\item A conscientização e disseminação dos resultados do projeto para a comunidade acadêmica e profissional, visando incentivar a adoção de técnicas de PPML e promover a importância da privacidade preservada em aplicações de aprendizado de máquina.
\end{itemize}

Espera-se que esses resultados contribuam para o avanço do conhecimento em PPML, forneçam insights valiosos sobre a preservação da privacidade em ambientes de aprendizado de máquina e incentivem a aplicação prática dessas técnicas em diferentes setores, garantindo a proteção dos dados pessoais dos indivíduos.

\section{Plano de trabalho - Maria Eduarda Carvalho Santos | }
\subsection{ Adequação, incluindo resumo, objetivos, Plano de Trabalho, problema de
pesquisa, Justificativa adequação plano-projeto-IC}
\subsection{Metodologia para alcance dos objetivos, com recursos materiais e infrastrutura |}
\subsection{Bibliografia}

\section{Plano de trabalho - Thaís Fernanda de Castro Garcia | }
\subsection{ Adequação, incluindo resumo, objetivos, Plano de Trabalho, problema de
pesquisa, Justificativa adequação plano-projeto-IC}
\subsection{Metodologia para alcance dos objetivos, com recursos materiais e infrastrutura |}
\subsection{Bibliografia}

\section{Plano de trabalho - Ualiton Ventura da Silva  | }
\subsection{ Adequação, incluindo resumo, objetivos, Plano de Trabalho, problema de
pesquisa, Justificativa adequação plano-projeto-IC}
\subsection{Metodologia para alcance dos objetivos, com recursos materiais e infrastrutura |}
\subsection{Bibliografia}

\section{Conclusão}
Neste projeto de iniciação científica, propomos um estudo aprofundado sobre as técnicas de Privacy Preserving Machine Learning (PPML), com o objetivo de compreender seus fundamentos teóricos e aplicar essas técnicas em uma prova de conceito. A preservação da privacidade em ambientes de aprendizado de máquina é uma área de pesquisa em crescimento, e os resultados obtidos neste projeto contribuirão para o avanço do conhecimento nesse campo. Através da análise dos resultados e da disseminação dos conhecimentos adquiridos, esperamos promover a adoção de técnicas de PPML e melhorar a proteção da privacidade dos indivíduos em aplicações de aprendizado de máquina.



\bibliographystyle{plainnat}
\bibliography{references}   

\end{document}
